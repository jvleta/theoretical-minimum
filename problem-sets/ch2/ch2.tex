\documentclass[10pt]{article}
 
\usepackage[margin=1in]{geometry} 
\usepackage{amsmath,amsthm,amssymb, graphicx, multicol, array}
\usepackage{physics}

\newcommand{\N}{\mathbb{N}}
\newcommand{\Z}{\mathbb{Z}}
 
\newenvironment{problem}[2][Problem]{\begin{trivlist}
\item[\hskip \labelsep {\bfseries #1}\hskip \labelsep {\bfseries #2.}]}{\end{trivlist}}

\begin{document}

\title{Chapter 2 Problems}
\author{Jimmy Leta\\
  Quantum Mechanics: The Theoretical Minimum}
\maketitle

\begin{problem}{2.1}
Prove that the vector $\ket{r}$ in Eq. 2.5 is orthoganl to vector $\ket{l}$ in Eq. 2.6.
\end{problem}

hi
%

\begin{problem}{2.2}
Prove that $\ket{i}$ and $\ket{o}$ satisfy all of the conditions in Eqs. 2.7, 2.8, and 2.9. Are they unique in that respect?
\end{problem}

\begin{proof}[Solution]
  hi
\end{proof}

\begin{problem}{2.3}
For the moment, forget that Eq.2.10 give us working definitions for $\ket{i}$ and $\ket{o}$
in terms of $\ket{u}$ and $\ket{d}$, and assume the components $\alpha$, $\beta$, $\gamma$, and
$\delta$ are unknown.
\begin{align*}
  \ket{i} & = \alpha \ket{u} + \beta\ket{d}   \\
  \ket{o} & = \gamma \ket{u} + \delta\ket{d}.
\end{align*}

\begin{enumerate}
  \item[a)] Use Eqs. 2.8 to show that
        \begin{align*}
          \alpha^*\alpha = \beta^*\beta = \gamma^*\gamma = \delta^*\delta = \frac{1}{2}
        \end{align*}
        \begin{proof}[Solution]
        \end{proof}

  \item[b)] Use the above result and Eq. 2.9 to show that
        \begin{align*}
          \alpha^*\beta = \alpha\beta^* = \gamma^*\delta = \gamma\delta^* = 0
        \end{align*}
        \begin{proof}[Solution]
        \end{proof}

  \item[c)] Show that $\alpha^*\beta$ and $\gamma^*\delta$ must each be pure maginary.
        \begin{proof}[Solution]
        \end{proof}
\end{enumerate}

\end{problem}

\end{document}
